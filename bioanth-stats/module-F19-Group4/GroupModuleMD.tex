\documentclass[]{article}
\usepackage{lmodern}
\usepackage{amssymb,amsmath}
\usepackage{ifxetex,ifluatex}
\usepackage{fixltx2e} % provides \textsubscript
\ifnum 0\ifxetex 1\fi\ifluatex 1\fi=0 % if pdftex
  \usepackage[T1]{fontenc}
  \usepackage[utf8]{inputenc}
\else % if luatex or xelatex
  \ifxetex
    \usepackage{mathspec}
  \else
    \usepackage{fontspec}
  \fi
  \defaultfontfeatures{Ligatures=TeX,Scale=MatchLowercase}
\fi
% use upquote if available, for straight quotes in verbatim environments
\IfFileExists{upquote.sty}{\usepackage{upquote}}{}
% use microtype if available
\IfFileExists{microtype.sty}{%
\usepackage{microtype}
\UseMicrotypeSet[protrusion]{basicmath} % disable protrusion for tt fonts
}{}
\usepackage[margin=1in]{geometry}
\usepackage{hyperref}
\hypersetup{unicode=true,
            pdftitle={Group Module},
            pdfauthor={Clint Lockwood,},
            pdfborder={0 0 0},
            breaklinks=true}
\urlstyle{same}  % don't use monospace font for urls
\usepackage{color}
\usepackage{fancyvrb}
\newcommand{\VerbBar}{|}
\newcommand{\VERB}{\Verb[commandchars=\\\{\}]}
\DefineVerbatimEnvironment{Highlighting}{Verbatim}{commandchars=\\\{\}}
% Add ',fontsize=\small' for more characters per line
\usepackage{framed}
\definecolor{shadecolor}{RGB}{248,248,248}
\newenvironment{Shaded}{\begin{snugshade}}{\end{snugshade}}
\newcommand{\AlertTok}[1]{\textcolor[rgb]{0.94,0.16,0.16}{#1}}
\newcommand{\AnnotationTok}[1]{\textcolor[rgb]{0.56,0.35,0.01}{\textbf{\textit{#1}}}}
\newcommand{\AttributeTok}[1]{\textcolor[rgb]{0.77,0.63,0.00}{#1}}
\newcommand{\BaseNTok}[1]{\textcolor[rgb]{0.00,0.00,0.81}{#1}}
\newcommand{\BuiltInTok}[1]{#1}
\newcommand{\CharTok}[1]{\textcolor[rgb]{0.31,0.60,0.02}{#1}}
\newcommand{\CommentTok}[1]{\textcolor[rgb]{0.56,0.35,0.01}{\textit{#1}}}
\newcommand{\CommentVarTok}[1]{\textcolor[rgb]{0.56,0.35,0.01}{\textbf{\textit{#1}}}}
\newcommand{\ConstantTok}[1]{\textcolor[rgb]{0.00,0.00,0.00}{#1}}
\newcommand{\ControlFlowTok}[1]{\textcolor[rgb]{0.13,0.29,0.53}{\textbf{#1}}}
\newcommand{\DataTypeTok}[1]{\textcolor[rgb]{0.13,0.29,0.53}{#1}}
\newcommand{\DecValTok}[1]{\textcolor[rgb]{0.00,0.00,0.81}{#1}}
\newcommand{\DocumentationTok}[1]{\textcolor[rgb]{0.56,0.35,0.01}{\textbf{\textit{#1}}}}
\newcommand{\ErrorTok}[1]{\textcolor[rgb]{0.64,0.00,0.00}{\textbf{#1}}}
\newcommand{\ExtensionTok}[1]{#1}
\newcommand{\FloatTok}[1]{\textcolor[rgb]{0.00,0.00,0.81}{#1}}
\newcommand{\FunctionTok}[1]{\textcolor[rgb]{0.00,0.00,0.00}{#1}}
\newcommand{\ImportTok}[1]{#1}
\newcommand{\InformationTok}[1]{\textcolor[rgb]{0.56,0.35,0.01}{\textbf{\textit{#1}}}}
\newcommand{\KeywordTok}[1]{\textcolor[rgb]{0.13,0.29,0.53}{\textbf{#1}}}
\newcommand{\NormalTok}[1]{#1}
\newcommand{\OperatorTok}[1]{\textcolor[rgb]{0.81,0.36,0.00}{\textbf{#1}}}
\newcommand{\OtherTok}[1]{\textcolor[rgb]{0.56,0.35,0.01}{#1}}
\newcommand{\PreprocessorTok}[1]{\textcolor[rgb]{0.56,0.35,0.01}{\textit{#1}}}
\newcommand{\RegionMarkerTok}[1]{#1}
\newcommand{\SpecialCharTok}[1]{\textcolor[rgb]{0.00,0.00,0.00}{#1}}
\newcommand{\SpecialStringTok}[1]{\textcolor[rgb]{0.31,0.60,0.02}{#1}}
\newcommand{\StringTok}[1]{\textcolor[rgb]{0.31,0.60,0.02}{#1}}
\newcommand{\VariableTok}[1]{\textcolor[rgb]{0.00,0.00,0.00}{#1}}
\newcommand{\VerbatimStringTok}[1]{\textcolor[rgb]{0.31,0.60,0.02}{#1}}
\newcommand{\WarningTok}[1]{\textcolor[rgb]{0.56,0.35,0.01}{\textbf{\textit{#1}}}}
\usepackage{graphicx,grffile}
\makeatletter
\def\maxwidth{\ifdim\Gin@nat@width>\linewidth\linewidth\else\Gin@nat@width\fi}
\def\maxheight{\ifdim\Gin@nat@height>\textheight\textheight\else\Gin@nat@height\fi}
\makeatother
% Scale images if necessary, so that they will not overflow the page
% margins by default, and it is still possible to overwrite the defaults
% using explicit options in \includegraphics[width, height, ...]{}
\setkeys{Gin}{width=\maxwidth,height=\maxheight,keepaspectratio}
\IfFileExists{parskip.sty}{%
\usepackage{parskip}
}{% else
\setlength{\parindent}{0pt}
\setlength{\parskip}{6pt plus 2pt minus 1pt}
}
\setlength{\emergencystretch}{3em}  % prevent overfull lines
\providecommand{\tightlist}{%
  \setlength{\itemsep}{0pt}\setlength{\parskip}{0pt}}
\setcounter{secnumdepth}{0}
% Redefines (sub)paragraphs to behave more like sections
\ifx\paragraph\undefined\else
\let\oldparagraph\paragraph
\renewcommand{\paragraph}[1]{\oldparagraph{#1}\mbox{}}
\fi
\ifx\subparagraph\undefined\else
\let\oldsubparagraph\subparagraph
\renewcommand{\subparagraph}[1]{\oldsubparagraph{#1}\mbox{}}
\fi

%%% Use protect on footnotes to avoid problems with footnotes in titles
\let\rmarkdownfootnote\footnote%
\def\footnote{\protect\rmarkdownfootnote}

%%% Change title format to be more compact
\usepackage{titling}

% Create subtitle command for use in maketitle
\providecommand{\subtitle}[1]{
  \posttitle{
    \begin{center}\large#1\end{center}
    }
}

\setlength{\droptitle}{-2em}

  \title{Group Module}
    \pretitle{\vspace{\droptitle}\centering\huge}
  \posttitle{\par}
    \author{Clint Lockwood,}
    \preauthor{\centering\large\emph}
  \postauthor{\par}
      \predate{\centering\large\emph}
  \postdate{\par}
    \date{11/22/2019}


\begin{document}
\maketitle

{
\setcounter{tocdepth}{2}
\tableofcontents
}
\begin{center}\rule{0.5\linewidth}{\linethickness}\end{center}

Within GIS there are multiple different tools to help you understand
data that you collect about different landscapes. From previous modules
you would have seen how to create maps to display your statistical
analysis, or have seen how to identify different clusters and analyze
significant hot and cold spots. Here we will explore how to take Lidar
data files to create maps that help you analyze the landscape and
interpret it to answer important questions. Lidar stands for Light
Detection and Ranging. It is a surveying/remote sensing method that
measures distance to a target. It does this by illuminating the target
with laser light and measuring the reflected light with a sensor.
Differences in laser return times and wavelengths can then be used to
make digital 3-D representations. It is commonly used to make
high-resolution maps, with applications in fields such as archaeology,
geography, and laser guidance. By being able to analyze landscapes using
Lidar, we can answer more important/pertinent questions to current
issues. One example of this is Boston's pledge to being carbon neutral
by 2050. While this falls under being a civil engineering problem, its
an excellent example of a different but important application of GIS,
specifically Lidar uses. One way that Boston can push towards being
carbon neutral would be by switching to renewable energy sources like
solar, wind, and hydrologic power. The best places to utilize solar
power, for example, would be places that have constant uninterrupted
exposure to sunlight. Using Lidar data to map out the rooftops of
Boston, we would be able to determine how much solar radiation we can
obtain. You would accomplish this by mapping out the rooftops and
overlaying a solar radiation layer (another built in-tool). This would
in turn help to determine if there would be enough watt hours over the
course of a year for it to be beneficial to begin transitioning the
majority, if not all, of Boston's rooftops to being solar paneled. info
\& data download here:
\url{https://drive.google.com/file/d/15Tdl4O4dje-7dQgGUciQpTgUza8LUo7Q/view}

\#Import LIDAR data \#\#Download packages ``lidR'' and ``raster''

\begin{Shaded}
\begin{Highlighting}[]
\CommentTok{#install.packages("lidR")}
\KeywordTok{library}\NormalTok{(lidR) }\CommentTok{#this one for sure}
\KeywordTok{library}\NormalTok{(sf)}
\KeywordTok{library}\NormalTok{(ggplot2)}
\CommentTok{#library(raster)}
\CommentTok{#ibrary(rgeos)}
\CommentTok{#library(mapview)}
\end{Highlighting}
\end{Shaded}

\#Get Raw LIDAR data into R \#\#raw LIDAR data The way LIDAR works is by
emitting laser pulses to reflect from objects both on and above the
ground surface: vegetation, buildings, bridges, etc. Any emitted laser
pulse that encounters multiple reflection surfaces as it travels toward
the ground is split into as many returns as there are reflective
surfaces.

\begin{itemize}
\item
  The first returned laser pulse is the most significant return and will
  be associated with the highest feature in the landscape like a treetop
  or the top of a building. The first return can also represent the
  ground, in which case only one return will be detected by the lidar
  system.
\item
  Multiple returns are capable of detecting the elevations of several
  objects within the laser footprint of an outgoing laser pulse. The
  intermediate returns, in general, are used for vegetation structure,
  and the last return for bare-earth terrain models.
\item
  The last return will not always be from a ground return. For example,
  consider a case where a pulse hits a thick branch on its way to the
  ground and the pulse does not actually reach the ground. In this case,
  the last return is not from the ground but from the branch that
  reflected the entire laser pulse.
\end{itemize}

\#\#lidR package function**

\begin{Shaded}
\begin{Highlighting}[]
\KeywordTok{setwd}\NormalTok{(}\StringTok{"C:/Users/clint/Desktop/Lab4_Data"}\NormalTok{)}
\NormalTok{las1 <-}\StringTok{ }\KeywordTok{readLAS}\NormalTok{(}\StringTok{"USGS_LPC_MA_Sndy_CMPG_2013_19TCG255905_LAS_2015.las"}\NormalTok{)}
\NormalTok{las2 <-}\StringTok{ }\KeywordTok{readLAS}\NormalTok{(}\StringTok{"USGS_LPC_MA_Sndy_CMPG_2013_19TCG270905_LAS_2015.las"}\NormalTok{)}
\CommentTok{#nk:}
\CommentTok{#las1 <- readLAS("~/Downloads/Lab4_Data/USGS_LPC_MA_Sndy_CMPG_2013_19TCG255905_LAS_2015/USGS_LPC_MA_Sndy_CMPG_2013_19TCG255905_LAS_2015.las")}
\CommentTok{#las2 <- readLAS("~/Downloads/Lab4_Data/USGS_LPC_MA_Sndy_CMPG_2013_19TCG270905_LAS_2015/USGS_LPC_MA_Sndy_CMPG_2013_19TCG270905_LAS_2015.las")}

\CommentTok{# Uses the raster package to combine raster images}
\NormalTok{las <-}\StringTok{ }\KeywordTok{rbind}\NormalTok{(las1, las2)}
\KeywordTok{plot}\NormalTok{(las) }\CommentTok{#30 seconds}
\end{Highlighting}
\end{Shaded}

\#Transition LIDAR data to DEM*

\#What is DEM? Why do we need it?* DEM stands for Digital Elevation
Model. This is a 3D CG representation of a terrain's surface created
from a terrain's elevation data. Things like power lines, buildings and
towers or trees and other vegetation aren't included in the model. It
filters out what are known as ``non-ground points''. Another way to
think of it is an image containing only the topography from the ground

\begin{Shaded}
\begin{Highlighting}[]
\NormalTok{dem <-}\StringTok{  }\KeywordTok{grid_terrain}\NormalTok{(las, }\DataTypeTok{algorithm =} \KeywordTok{tin}\NormalTok{())}
\KeywordTok{plot}\NormalTok{(dem) }\CommentTok{# 1.5 min}
\end{Highlighting}
\end{Shaded}

\#Convert to DSM + create hillshade model \#\#What is DSM? Why do we
need it? DSM stands for Digital Surface Model. Unlike the DEM, the DSM
DOES include man-made/built and natural points of reference into its
representation. Another way to think of it is an image including the
first return information. \#\#What is hillshade? Why do we need it? How
is it made?

\begin{Shaded}
\begin{Highlighting}[]
\KeywordTok{proj4string}\NormalTok{(las) <-}\KeywordTok{CRS}\NormalTok{(}\StringTok{"+proj=utm +zone=19 +datum=NAD83"}\NormalTok{) }\CommentTok{#assigning coord ref system}
\NormalTok{dsm1 <-}\StringTok{ }\KeywordTok{grid_canopy}\NormalTok{(las, }\DataTypeTok{res =} \DecValTok{1}\NormalTok{, }\KeywordTok{dsmtin}\NormalTok{())}
\NormalTok{col <-}\StringTok{ }\KeywordTok{height.colors}\NormalTok{(}\DecValTok{100}\NormalTok{) }\CommentTok{#how many colors to use, the more colors the better resolution in the plot}
\KeywordTok{plot}\NormalTok{(dsm1, }\DataTypeTok{col =}\NormalTok{ col)}
\end{Highlighting}
\end{Shaded}

\#Elaborate Hillshade. What is it? How is is made?*

\begin{Shaded}
\begin{Highlighting}[]
\NormalTok{slope  <-}\StringTok{ }\KeywordTok{terrain}\NormalTok{(dsm1, }\DataTypeTok{opt=}\StringTok{'slope'}\NormalTok{)}
\NormalTok{aspect <-}\StringTok{ }\KeywordTok{terrain}\NormalTok{(dsm1, }\DataTypeTok{opt=}\StringTok{'aspect'}\NormalTok{)}
\NormalTok{hill <-}\StringTok{ }\KeywordTok{hillShade}\NormalTok{(slope, aspect, }\DecValTok{45}\NormalTok{, }\DecValTok{120}\NormalTok{)}
\KeywordTok{plot}\NormalTok{(hill, }\DataTypeTok{col=}\KeywordTok{grey}\NormalTok{(}\DecValTok{0}\OperatorTok{:}\DecValTok{100}\OperatorTok{/}\DecValTok{100}\NormalTok{), }\DataTypeTok{legend=}\OtherTok{FALSE}\NormalTok{, }\DataTypeTok{main=}\StringTok{'Boston University'}\NormalTok{)}
\KeywordTok{plot}\NormalTok{(dsm1, }\DataTypeTok{col=}\KeywordTok{rainbow}\NormalTok{(}\DecValTok{25}\NormalTok{, }\DataTypeTok{alpha=}\FloatTok{0.35}\NormalTok{),}\DataTypeTok{add=}\OtherTok{TRUE}\NormalTok{) }\CommentTok{# 2 min}
\end{Highlighting}
\end{Shaded}

\#Add Boston Univeristy rooftop shapefile \#Elaborate shapefile
description*

\begin{Shaded}
\begin{Highlighting}[]
\NormalTok{BU_buildings_shp <-}\StringTok{ }\KeywordTok{st_read}\NormalTok{(}\StringTok{"building_structures_bu.shp"}\NormalTok{)}
\NormalTok{BU_buildings <-}\StringTok{ }\KeywordTok{ggplot}\NormalTok{() }\OperatorTok{+}\StringTok{ }
\StringTok{  }\KeywordTok{geom_sf}\NormalTok{(}\DataTypeTok{data =}\NormalTok{ BU_buildings_shp, }\DataTypeTok{size =} \FloatTok{0.5}\NormalTok{, }\DataTypeTok{color =} \StringTok{"black"}\NormalTok{, }\DataTypeTok{fill =} \StringTok{"cyan1"}\NormalTok{) }\OperatorTok{+}\StringTok{ }
\StringTok{  }\KeywordTok{ggtitle}\NormalTok{(}\StringTok{"BU Buildings"}\NormalTok{) }\OperatorTok{+}\StringTok{ }
\StringTok{  }\KeywordTok{coord_sf}\NormalTok{()}
\KeywordTok{plot}\NormalTok{(BU_buildings) }\CommentTok{# 5 sec}
\end{Highlighting}
\end{Shaded}

\#Plot Building Vector File on Raster

\begin{Shaded}
\begin{Highlighting}[]
\CommentTok{#This projects the BU rooftop shape file correctly with the hillshade raster image}
\NormalTok{dsm2 <-}\StringTok{ }\KeywordTok{projectRaster}\NormalTok{(dsm1, }\DataTypeTok{crs =} \KeywordTok{crs}\NormalTok{(BU_buildings))}
\NormalTok{dsm3 <-}\StringTok{ }\KeywordTok{crop}\NormalTok{(dsm2, BU_buildings) }\CommentTok{#there are 16 different options instead of "2" here, not sure which one we want}

\NormalTok{hill2 <-}\StringTok{ }\KeywordTok{projectRaster}\NormalTok{(hill, }\DataTypeTok{crs =} \KeywordTok{crs}\NormalTok{(BU_buildings))}
\NormalTok{hill3 <-}\StringTok{ }\KeywordTok{crop}\NormalTok{(hill2, BU_buildings) }\CommentTok{#there are 16 different options instead of "2" here, not sure which one we want}

\KeywordTok{plot}\NormalTok{(hill3, }\DataTypeTok{col=}\KeywordTok{grey}\NormalTok{(}\DecValTok{0}\OperatorTok{:}\DecValTok{100}\OperatorTok{/}\DecValTok{100}\NormalTok{), }\DataTypeTok{legend=}\OtherTok{FALSE}\NormalTok{, }\DataTypeTok{main =}\StringTok{'Boston University'}\NormalTok{)}
\KeywordTok{plot}\NormalTok{(BU_buildings,}\DataTypeTok{add=}\OtherTok{TRUE}\NormalTok{)}
\KeywordTok{plot}\NormalTok{(dsm3,}\DataTypeTok{col=}\KeywordTok{rainbow}\NormalTok{(}\DecValTok{25}\NormalTok{, }\DataTypeTok{alpha=}\FloatTok{0.35}\NormalTok{),}\DataTypeTok{add=}\OtherTok{TRUE}\NormalTok{) }\CommentTok{# 30 sec}
\end{Highlighting}
\end{Shaded}

\#Outdated script \#Get a processed DSM file into R

\begin{Shaded}
\begin{Highlighting}[]
\NormalTok{dsm <-}\StringTok{ }\KeywordTok{raster}\NormalTok{(}\StringTok{"bu_dsm.tif"}\NormalTok{)}
\KeywordTok{plot}\NormalTok{(dsm)}
\end{Highlighting}
\end{Shaded}

\#Create a hillshade from DSM

\begin{Shaded}
\begin{Highlighting}[]
\NormalTok{slope  <-}\StringTok{ }\KeywordTok{terrain}\NormalTok{(dsm, }\DataTypeTok{opt=}\StringTok{'slope'}\NormalTok{)}
\NormalTok{aspect <-}\StringTok{ }\KeywordTok{terrain}\NormalTok{(dsm, }\DataTypeTok{opt=}\StringTok{'aspect'}\NormalTok{)}
\NormalTok{hill <-}\StringTok{ }\KeywordTok{hillShade}\NormalTok{(slope, aspect, }\DecValTok{45}\NormalTok{, }\DecValTok{120}\NormalTok{)}
\KeywordTok{quartz}\NormalTok{()}
\KeywordTok{plot}\NormalTok{(hill, }\DataTypeTok{col=}\KeywordTok{grey}\NormalTok{(}\DecValTok{0}\OperatorTok{:}\DecValTok{100}\OperatorTok{/}\DecValTok{100}\NormalTok{), }\DataTypeTok{legend=}\OtherTok{FALSE}\NormalTok{, }\DataTypeTok{main=}\StringTok{'Boston University'}\NormalTok{)}
\KeywordTok{plot}\NormalTok{(dsm1, }\DataTypeTok{col=}\KeywordTok{rainbow}\NormalTok{(}\DecValTok{25}\NormalTok{, }\DataTypeTok{alpha=}\FloatTok{0.35}\NormalTok{), }\DataTypeTok{add=}\OtherTok{TRUE}\NormalTok{)}
\end{Highlighting}
\end{Shaded}

\#unaligned raster + vector file

\begin{Shaded}
\begin{Highlighting}[]
\KeywordTok{plot}\NormalTok{(hill, }\DataTypeTok{col=}\KeywordTok{grey}\NormalTok{(}\DecValTok{0}\OperatorTok{:}\DecValTok{100}\OperatorTok{/}\DecValTok{100}\NormalTok{), }\DataTypeTok{legend=}\OtherTok{FALSE}\NormalTok{, }\DataTypeTok{main =}\StringTok{'Boston University'}\NormalTok{)}
\KeywordTok{plot}\NormalTok{(BU_buildings,}
     \DataTypeTok{add =} \OtherTok{TRUE}\NormalTok{)}
\KeywordTok{plot}\NormalTok{(dsm1, }\DataTypeTok{col=}\KeywordTok{rainbow}\NormalTok{(}\DecValTok{25}\NormalTok{, }\DataTypeTok{alpha=}\FloatTok{0.35}\NormalTok{), }\DataTypeTok{add=}\OtherTok{TRUE}\NormalTok{)}
\end{Highlighting}
\end{Shaded}


\end{document}
